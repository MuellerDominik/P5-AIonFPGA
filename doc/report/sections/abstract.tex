\vspace*{\fill}
\begin{abstract}
% Motivation
\noindent In a world of self-driving cars and automated quality control in manufacturing, real-time image classification is becoming increasingly important.
Artificial intelligence, and deep learning in particular, are achieving excellent classification accuracies, but there are some challenges.
% Problem statement
For one thing, high-resolution image acquisition systems require a lot of processing power.
For another, a large labeled dataset of training data is required to train deep convolutional neural networks.
% Approach
A solution for the former is to use field-programmable gate arrays as hardware accelerators.
The focus of this project, however, is to collect a labeled dataset.
For this purpose, an industrial camera with a frame rate of \SI{200}{fps} is used to acquire images of 22 different throwing objects.
To simplify this task, a camera throw detection mechanism is implemented and various Python scripts are used.
% Results
The developed software uses an image change detection algorithm to detect individual throws.
The result is a fully labeled dataset that consists of more than \num{15000} usable images with at least 480 images of each object.
% Conclusions
Developing software to support the image acquisition proved to be extremely useful.
It made it possible to collect the complete dataset in only two days.

% from https://users.ece.cmu.edu/~koopman/essays/abstract.html

\vspace{3em}
\begin{tabular}{>{\bfseries}ll}
  Team members & Nico Canzani, Dominik M\"uller \\
  Keywords & Artificial Intelligence, FPGA, Labeled Data, Industrial Camera
\end{tabular}
\end{abstract}
\vspace*{\fill}
