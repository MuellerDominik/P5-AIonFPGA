\vspace*{\fill}
\begin{abstract}
%% Motivation
% AI is the solution
\noindent In a world of self-driving cars and automated quality control in manufacturing, real-time image classification is becoming increasingly important.
Artificial intelligence, and deep learning in particular, are achieving excellent classification accuracies, but there are some challenges.

%% Problem statement
% Collect enought data to train the CNN
% New problem: needs dataset to train + needs a lot of resources, thus implemented on hardware
For one thing, high-resolution image acquisition systems require a lot of processing power.
For another, a large labeled dataset of training data is required to train deep convolutional neural networks.
% The problems are, the required processing power and a large labeled dataset of training data.

%% Approach
% industrial camera +
% manually too error prone and time consuming => software
A solution for the former is to use field-programmable gate arrays as hardware accelerators.
The focus of this project, however, is to collect a labeled dataset.
For this purpose, an industrial camera with a frame rate of \SI{200}{fps} is used to acquire images of 22 different throwing objects.
To simplify this task, a camera throw detection mechanism is implemented and various Python scripts are used.
% When done manually, the camera has to be started and stopped for each throw.
% This likely creates many empty frames at the beginning and at the end of the capture.
% Those empty and therefore invalid frames must be removed to avoid errors during the training of the CNN later on.
% Furthermore, each valid frame must be labeled.
% This procedure is very error-prone if it is performed manually for each individual frame.
% To simplify this task, a camera throw detection mechanism is implemented and various Python scripts are used.

%% Results
% software created (throw detection)
% over 15000 pictures collected
% The developed software uses an image change detection algorithm to detect an individual throw.
The developed software uses an image change detection algorithm to detect individual throws.
% % The result is a software that uses an image change detection algorithm to detect individual throws.
% The dataset is fully labeled and consists of more than \num{15000} usable images with at least 480 images of each object.
The result is a fully labeled dataset that consists of more than \num{15000} usable images with at least 480 images of each object.

%% Conclusions
% Software showed to be extremely helpful
% Developing software to aid the collection of the dataset proved to be extremely useful.
% The development of software to support the data collection proved to be extremely useful.
% The development of software to support the image acquisition proved to be extremely useful.
Developing software to support the image acquisition proved to be extremely useful.
% This made it possible to collect the complete dataset in only two days.
It made it possible to collect the complete dataset in only two days.
% With the help of the throw detection mechanism and is was

% from https://users.ece.cmu.edu/~koopman/essays/abstract.html

\vspace{3em}
\begin{tabular}{>{\bfseries}ll}
  Team members & Nico Canzani, Dominik M\"uller \\
  Keywords & Artificial Intelligence, FPGA, Labeled Data, Industrial Camera
\end{tabular}
\end{abstract}
\vspace*{\fill}
