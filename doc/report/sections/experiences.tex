\section{Experiences}
\label{sec:experiences}

The FHNW still has very little experience in the field of artificial intelligence on FPGAs.
Therefore, the first part of the project consisted only of gathering information.
Due to the fact that Xilinx hardware was provided, research could be focused on this product.
%Aside from the hardware definition the field of AI was still a very big world, fairly unknown to us.
Apart from the hardware definition, the field of AI was still quite unknown to us.
With the help of YouTube videos, other internet resources and books we slowly worked our way into the unknown zone. 
Over time, the topic became more accessible to us, but there are still questions to which we do not yet have the right answers.
For example, how good should the picture quality be?
May pictures be used if the object is only partially in the picture?
How many pictures of an individual object are enough?
These questions are usually answered by experience.
%From our lack of experience we have marked cut-off images in the database, kept the quality as high as possible and informed ourselves about the sizes of other datasets on the Internet.
Due to our lack of experience we marked images with only partially visible objects in the database, kept the quality as high as possible and informed ourselves about the size of other datasets on the internet.

%Choosing a suitable camera without much experience is not trivial.
Choosing a suitable camera without a lot of experience is not trivial.
We had to learn this the hard way.
%We also learned the hard way when choosing a camera with too little experience.
%We also had to learn the hard way when choosing the camera.
%The plug-and-play version of a webcam sounded attractive, but turned out to be useless for moving objects.
The plug-and-play version of a webcam sounded attractive, but proved to be useless for moving objects.

The Python scripts turned out to be a reasonable investment.
%If we had manually started and stopped the camera for each throw, with the Baumer application, countless empty images would have to be removed before and after the throw.
If we had manually started and stopped the camera for each throw, countless empty images before and after the throw would have had to be removed.
Thanks to the real-time throw detection mechanism, we have saved ourselves this work.
%The additional effort for configuring the camera via script also benefits us in project 6.
%The additional effort for configuring the camera via script will also benefits us during our thesis.
%The additional effort for the configuration of the camera by a program will also benefit us during our diploma thesis.
The additional effort of configuring the camera with a program will also benefit us during our thesis.

At the end of this project, the results are very pleasing.
%We are in possession of a throwing booth, a dataset with more than \num{15000} images, as well as knowledge about AI.
We are in possession of a throwing booth, a dataset with more than \num{15000} labeled images and newly gained knowledge about AI.
