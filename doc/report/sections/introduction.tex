\section{Introduction}
\label{sec:introduction}
\todo{Nico: Letzten zwei Abschnitte ausbauen, Text korrigieren}
%Die technische Welt befindet sich in einem ständigen Wandel. Eine der gewichtigsten Neuerungen ist die Künstliche Intelligenz, oder kurz KI genannt.
%Künstliche Intelligenz ist der Grundbaustein diverser Grossunternehmen, wie beispielsweise Google oder Facebook.
%Auch bei den Privatpersonen hat die KI Einzug gehalten.
%Fei-Fei Li, Professor of Computer Science at Stanford University sagte mal:
% 	Zitat
The technical world is in a constant state of change. One of the most important innovations is artificial intelligence, or AI for short.
Artificial intelligence is the basic building block of various large companies, such as Google or Facebook.
AI has also found its way into the world of private individuals.
Fei-Fei Li, Professor of Computer Science at Stanford University said once:
\begin{quote}
	"I imagine a world in which AI is going to make us work more productively, live longer, and have cleaner energy."\cite{QuotesFuture}
\end{quote}
This scenario has already begun. As an example of what artificial intelligence is capable of, the above paragraph was translated from a German text using the artificial intelligence of DeepL. The result is impressive.
However, AI is very versatile. In medicine, it is used to make diagnoses, cars should be able to drive independently or banks can recognize unauthorized use of credit cards\cite{ArtificialIntelligenceAModernApproach}. \todo{zweiter Autor steht noch N.P. in den Quellen}
The areas of application can be extended as required.

Another challenge in today's world is the flood of data. More and more data should be processed in the shortest possible time. This makes fast hardware indispensable.

Project 5, AI High-Performance Solution, covers two topics, artificial intelligence and fast data processing.
A demo object is to be developed in which artificial intelligence is implemented on an FPGA.
The demo object will be shown at exhibitions and represents the FHNW.
It is a throwing booth, which should be able to detect and recognize flying objects.
The booth should be able to operate as a standalone.

An Ultra96-V2 Development Board serves as hardware.
The MPSoC used there, an UltraScale+ MPSoC ZU3EG, has arm-based microprocessors and an FPGA.

The first step is to create the basic conditions for designing an AI.
This is, on the one hand, the litter status and, on the other hand, a data set. A data set is needed to train a neural network.
In addition, project 5 includes an introduction to the field of artificial intelligence.
In project 6, based on project 5, artificial intelligence will be implemented on the Ultra96 board.

This technical report describes the theoretical and technical principles needed to implement the project.