\subsection{Database}
\label{subsec:database}

All images contain their respective labels in the file name.
This is useful for manually going through them, but not optimal for a program.
The program would have to crawl the directory containing the images and parse the file names.
This requires file system related operations, which are usually very slow.
The solution is to use a database that is designed to read and write such data.
For this reason, a MySQL database is used to store the labels, file names and other metadata.
Table \ref{tab:tab_frames_structure} lists all columns of the database table \texttt{aionfpga.frames}.

\begin{table}[hb]
  \caption{Structure of the database table \texttt{aionfpga.frames}}
  \label{tab:tab_frames_structure}
  \centering
  \begin{tabular}{llll}
    \toprule
    \textbf{Column} & \textbf{Type} & \textbf{Length} & \textbf{Description} \\
    \midrule
    id & \texttt{INT} &  & Sequence number (unique identifier) \\
    timestamp & \texttt{INT} &  & The Unix timestamp at the time of the throw \\
    throwid & \texttt{INT} &  & Throw sequence number \\
    frameid & \texttt{INT} &  & Frame sequence number within a throw \\
    frame & \texttt{VARCHAR} & 255 & The file name of the frame \\
    object & \texttt{VARCHAR} & 255 & The name of the object in the frame (label) \\
    framegood & \texttt{INT} &  & \texttt{0}: frame unusable | \texttt{1}: frame usable \\
    partial & \texttt{INT} &  & \texttt{0}: object fully visible | \texttt{1}: object partially visible \\
    \bottomrule
  \end{tabular}
\end{table}
