\subsubsection{Image Change Detection}
\label{subsubsec:image_change_detection}
\todo[inline]{Citations, vspace before listings}

This section describes the implementation of the image change detection.
The difference computation and the averaging among all the pixels is done for every frame except the first one ($\texttt{FID} \geq 1$).
The threshold comparison and the throw detection is done for every frame after the threshold has been determined (see section \ref{subsubsec:threshold}).

% --------------------------------
\paragraph{Difference Computation}
The absolute difference pixel matrix $D$ can be calculated using equation \ref{eq:abs_diff}.
The implementation uses the OpenCV function \texttt{cv::absdiff} to calculate the per-element absolute difference between the two pixel matrices and saves it into \texttt{cv\_abs}, as shown in appendix \ref{app:throw_detection} on line \ref{lst:ln:abs_diff}.
% https://docs.opencv.org/4.1.1/d2/de8/group__core__array.html#ga6fef31bc8c4071cbc114a758a2b79c14

\begin{equation}
  D = |I_\text{FID} - I_\text{FID-1}| \quad\quad \text{with} \quad\quad \text{FID} \geq 1
  \label{eq:abs_diff}
\end{equation}

\begin{tabular}{rl}
  $D =$ & absolute difference pixel matrix \\
  $I =$ & image pixel matrix \\
\end{tabular}

% --------------------------------------
\paragraph{Average among all the pixels}
The mean absolute difference $\overline{\text{MD}}$ can be calculated using equation \ref{eq:mean_diff}.
The implementation uses the OpenCV function \texttt{cv::sum} to return the sum of array elements for each channel independently, as shown in appendix \ref{app:throw_detection} on line \ref{lst:ln:mean_diff}.
Since the Baumer \texttt{BayerRG8} pixel format consists of only one channel, the zeroth element is used.
% https://docs.opencv.org/4.1.1/d2/de8/group__core__array.html#ga716e10a2dd9e228e4d3c95818f106722

\begin{equation}
  \overline{\text{MD}} = \frac{1}{w\cdot h} \cdot \sum\limits_{i=1}^h \sum\limits_{j=1}^w D_{i,j}
  \label{eq:mean_diff}
\end{equation}

\begin{tabular}{rl}
  $\text{MD} =$ & mean absolute difference \\
  $D =$ & absolute difference pixel matrix \\
  $w =$ & width of the image in px \\
  $h =$ & height of the image in px \\
\end{tabular}

% --------------------------------------------------
\paragraph{Threshold Comparison and Throw Detection}
Listing \ref{lst:throw_detection} shows the implementaion of the threshold comparison and the throw detection.
Due to the specific implementation, it is possible to detect a single frame change.
This means that the image changes between two consecutive frames and then stays the way it is (e.g. sudden change of the ambient light).
Such a glitch is not a valid throw and shall therefore be ignored (see line \ref{lst:ln:glitch_removal}).
Figure \ref{subfig:algorithm_example_1} shows that a valid throw of an object would create at least two changes between individual frames.
The first one by entering and the second one by leaving the frame.

\vspace{5pt}
\begin{lstlisting}[style=C++, caption={Throw detection and glitch removal}, label=lst:throw_detection]
  if (mean_diff >= threshold) {
      if (!throw_bgn) {
        throw_bgn_idx = frame_id;
        throw_bgn = true;
      }
  } else {
    if (throw_bgn) {
      throw_end_idx = frame_id;

      // Remove glitches (single frame changes)
      if ((throw_end_idx - throw_bgn_idx) == 1) {(*\label{lst:ln:glitch_removal}*)
        throw_bgn = false;
      } else {
        throw_end = true;
      }
    }
  }
\end{lstlisting}

As long as no throw is detected, the current Baumer \texttt{Buffer} object is released after each processed frame (see listing \ref{lst:buffer_release}).
Whenever a glitch is removed, the size of the Baumer image buffer (\texttt{BS}) is reduced by one.
This is due to the fact that no reference to past \texttt{Buffer} objects is kept.
Such a reference is necessary to release the respective buffer entry.
It would be easy to solve this by keeping track of at least one past \texttt{Buffer} object.
However, this is not necessary, provided that the buffer size is large enough.

\vspace{5pt}
% BUG: Should not be released directly but with a delay of one!
\begin{lstlisting}[style=C++, caption={Release of the filled Baumer \texttt{Buffer} object}, label=lst:buffer_release]
  if (!throw_bgn) {
    pBufferFilled->QueueBuffer();
  }
\end{lstlisting}
