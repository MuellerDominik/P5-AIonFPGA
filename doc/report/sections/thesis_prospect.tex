\section{Thesis Prospect}
\label{sec:thesis_prospect}
% BOX fabrication / net -- 
% CNN -- The CNN model is developed and trained in Python 3 using TensorFlow 2 and the images collected in project 5.
% parallel change detection algo.
% High-Performance Implementation --  The advantages of an FPGA flow into the development. Thus, special attention is paid to the performance of the system.
% OS / application
% Verification -- The accuracy and performance of the CNN model implemented on the FPGA is verified.

This project will be continued as a thesis in the next semester.
The throwing booth will be upgraded with the Fibox and a net construction.
The box can be manufactured according to the finished drawings, but the net construction has to be drawn first.
The net holder will most likely be manufactured with a 3D printer.

The CNN model is developed and trained in Python 3 using TensorFlow 2 with the dataset collected during this project.
%In order to be able to evaluate the changes of the images on the ARM, a parallelization will probably be necessary.
In order to be able to use the throw detection mechanism on the quad-core ARM-based processor, parallel computing will probably be necessary.
The reason for this is the lower clock speed of the ARM Cortex-A53 processors for the same amount of data.

As soon as this is realized, the next step is to implement the trained CNN model on the FPGA.
With sufficient time resources, an additional focus can be placed on speed optimizations.

At the same time a Linux-based operating system (OS) is put into operation on the MPSoC.
The application shows the recognized object to the user at the trade fair.
This function is implemented on the ARM Cortex-A53 as well.

To measure the quality of the developed system, the accuracy of the convolutional neural network is evaluated.
Special attention is paid to the differences between the implementation on the computer and the implementation on the embedded system.
