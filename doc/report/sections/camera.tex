\section{Camera}
\label{sec:camera}
\todo[inline]{Dominik / Nico}
\subsection{Webcam}
\label{subsec:webcam}
A simple webcam is not enough to capture fast moving objects with sufficient quality. 
There are several reasons for this.
Firstly, certain parameters, such as the exposure time, cannot be adjusted by the user.
This would be necessary to capture the contours of the object being thrown as detailed as possible.
Secondly, conventional webcams work with a rolling shutter.
Technically this means that fewer sensors are used. Therefore one sensor processes several pixels one after the other.

This results in the rolling shutter effect when the image changes.
On the one hand this effect is noticeable when the light flickers quickly.
For example the upper part of an image can be brighter than the lower part.
On the other hand there are problems with moving pictures.
The lower part of the image was taken at a later time, so the object appears distorted \cite{GlobalRollingShutter}.

To eliminate this problem, some cameras feature the global shutter. 
This takes all pixels at the same time and stores them. 
Thus no distortions in the image occur.

This can be shown well with an example.
In the figure subfig1 is a recording of a ball with rolling shutter. 
Figure subfig2 shows the same object, taken with a global shutter camera.
The distortion from the rolling shutter is clearly visible and distorts the effective object.

\todo[inline]{Bild Rolling shutter vs global shutter einfügen}

\subsection{Baumers industrial camera}
\label{subsec:baumer_cam}

\subsection{Lighting}
\label{subsec:Lighting}
\todo[inline]{mit Intensität, nach aussen immer schwächer}